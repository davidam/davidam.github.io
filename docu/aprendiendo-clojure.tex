% Created 2013-03-12 mar 02:20
\documentclass[11pt]{article}
\usepackage[utf8]{inputenc}
\usepackage[T1]{fontenc}
\usepackage{fixltx2e}
\usepackage{graphicx}
\usepackage{longtable}
\usepackage{float}
\usepackage{wrapfig}
\usepackage{soul}
\usepackage{textcomp}
\usepackage{marvosym}
\usepackage{wasysym}
\usepackage{latexsym}
\usepackage{amssymb}
\usepackage{hyperref}
\usepackage[spanish,spain]{babel}
\tolerance=1000
\providecommand{\alert}[1]{\textbf{#1}}

\title{Aprendiendo Clojure}
\author{David Arroyo Menéndez}
\date{\today}
\hypersetup{
  pdfkeywords={},
  pdfsubject={},
  pdfcreator={Emacs Org-mode version 7.8.11}}

\begin{document}

\maketitle

\setcounter{tocdepth}{3}
\tableofcontents
\vspace*{1cm}

Lisp es habitualmente experto y puntero en la creación de nuevas
características de lenguajes de programación, sin embargo, otros
lenguajes como Java suelen tener librerías que en ocasiones son útiles
para los programadores, con Clojure es posible utilizar las librerías
Java desde Lisp.

\section{Instalación y Primeros Pasos}
\label{sec-1}


Instalar Clojure en Debian es sencillo


\begin{verbatim}
sudo apt-get install clojure1.4
\end{verbatim}

También es fácil hacer un ``Hello World'' desde la típica gui de Java:


\begin{verbatim}
(javax.swing.JOptionPane/showMessageDialog nil "Hello World" )
\end{verbatim}

Si nuestras líneas clojure están en un fichero también podemos
interpretarlas desde bash. Pongamos que nuestro fichero transforma de
grados fahrenheit a celsius:


\begin{verbatim}
#! /usr/bin/env clojure
(def fahrenheit (first *command-line-args*))
(println (* 0.556 (- (read-string fahrenheit) 32)))
\end{verbatim}

En tal caso podemos ejecutar el siguiente comando:


\begin{verbatim}
$ ./fahrenheit2celsius.clj 5
-15.012
\end{verbatim}
\section{Próximo Día, profundizar android y clojure}
\label{sec-2}


\begin{itemize}
\item \href{https://github.com/technomancy/leiningen/blob/stable/doc/TUTORIAL.md}{https://github.com/technomancy/leiningen/blob/stable/doc/TUTORIAL.md}
\item \href{https://github.com/alexander-yakushev/lein-droid/wiki/Tutorial}{https://github.com/alexander-yakushev/lein-droid/wiki/Tutorial}
\end{itemize}

\end{document}
