% Created 2015-08-22 sáb 14:27
\documentclass[11pt]{article}
\usepackage[utf8]{inputenc}
\usepackage[T1]{fontenc}
\usepackage{fixltx2e}
\usepackage{graphicx}
\usepackage{longtable}
\usepackage{float}
\usepackage{wrapfig}
\usepackage{rotating}
\usepackage[normalem]{ulem}
\usepackage{amsmath}
\usepackage{textcomp}
\usepackage{marvosym}
\usepackage{wasysym}
\usepackage{amssymb}
\usepackage{capt-of}
\usepackage{hyperref}
\tolerance=1000
\usepackage[spanish]{babel}
\author{David Arroyo Menéndez}
\date{\today}
\title{Traduciendo Documentación en Texinfo con emacs y po4a}
\hypersetup{
 pdfkeywords={},
  pdfsubject={},
  pdfcreator={Emacs 24.3.1 (Org mode 8.3beta)}}
\begin{document}

\maketitle
\tableofcontents


\section{Instalación}
\label{sec-1}

Texinfo es el formato de marcas estándar para documentación de
GNU. Permite crear manuales en formato pdf, info, html, etc. desde un
solo fichero fuente .texi con las características avanzadas que
debería tener un buen manual.

Antes de familiarizarnos con los comandos del proceso de traducción
instalemos los paquetes debian que vamos a utilizar:

\begin{verbatim}
apt-get install texlive texinfo texi2html emacs gettext-el po4a
aspell-es dict dict-freedict-eng-spa dict-freedict-spa-eng
dictionary-el dict-gcide
\end{verbatim}

\section{Comandos}
\label{sec-2}

Si tenemos un fichero texinfo y queremos convertirlo a pdf se puede usar el siguiente comando:

\begin{verbatim}
texi2pdf myfile.texi
\end{verbatim}

El problema con texi2pdf es que a veces no ofrece buenos mensajes de depuración, para depurar es mejor hacerlo con:

\begin{verbatim}
makeinfo myfile.texi
\end{verbatim}

Además con este comando se crea el fichero .info

Si queremos crear un sitio web con nuestro documento podemos usar:

\begin{verbatim}
texi2html myfile.texi
\end{verbatim}

Si se desea traducir el fichero del inglés al español u otras lenguas es posible utilizar gettext gracias a po4a. Siguiendo estos pasos:

\begin{verbatim}
po4a-gettextize -f texinfo -m myfile.texi -p myfile.pot # Crea un fichero .pot a partir de un .texi
cp myfile.pot myfile.es.po
# Se traduce del inglés al español abriendo myfile.es.po con emacs en modo po-mode
po4a-translate -f texinfo -m myfile.texi -p myfile.es.po -l myfile.es.texi # Creamos myfile.es.texi, el texi traducido
\end{verbatim}

\section{Actualizando}
\label{sec-3}

Es probable que con el tiempo salgan nuevas actualizaciones del fichero .texi original, podemos actualizarnos con msgmerge del siguiente modo:

\begin{verbatim}
po4a-gettextize -f texinfo -m myupdated.texi -p myupdated.pot # Crea el fichero .pot actualizado a partir del nuevo .texi
msgmerge myold.es.po myupdated.pot > myupdated.es.po
\end{verbatim}

Las frases en las que haya habido variaciones aparecerán como fuzzy debiendo ser debidamente revisadas.

\section{GNU Emacs}
\label{sec-4}

Y ahora vemos algo de configuración del .emacs para hacer más fácil y divertida la traducción:

\begin{verbatim}
;;;; PO-MODE ;;;;
(require 'po-mode)
(defun iacute()
(interactive)
(insert "@'{@dotless{i}}"))
;; El atajo C-i insertará i acentuada, a la manera nativa de texinfo
(define-key po-subedit-mode-map "\C-i" 'iacute)

;; Debe cortar las líneas automáticamente
(add-hook 'po-subedit-mode-hook 'auto-fill-mode)

;;;; ASPELL ;;;;
;; Para que avise
(setq ispell-program-name "aspell"
ispell-extra-args '("--lang=es"))
(add-hook 'po-subedit-mode-hook
(lambda()
(flyspell-mode 1)))

;;;; Dictionary ;;;;
(global-set-key "\C-cs" 'dictionary-search)
(global-set-key "\C-cm" 'dictionary-match-words)
\end{verbatim}
\section{Algunos Criterios Personales de Traducción}
\label{sec-5}
Estos son algunos buenos acuerdos que sin tratar de ser completos sí
son útiles para otros hackers con quienes he compartido tiempo traduciendo.
\subsection{Capitalización}
\label{sec-5-1}
En títulos y subtítulos solo aparecerá en mayúscula la primera letra.
\subsection{Traducción de Código}
\label{sec-5-2}
Para evitar problemas, solo se traducirán comentarios y cadenas de
texto que se deban imprimir por pantalla. No variables, ni funciones.
\subsection{Impersonal}
\label{sec-5-3}
Las frases se escribirán de manera impersonal. Es decir, no se dirá
"prueba este código", sino "se va a probar este código"
\subsection{Una palabra dos idiomas}
\label{sec-5-4}
A veces ayuda a la comprensión lectora el aparte de traducir la
palabra dejarla en el idioma original. En este caso se dejará con la
etiqueta texi @samp.

\section{Licencia}
\label{sec-6}
Copyright (C) 2015  David Arroyo Menéndez
    Se permite copiar, distribuir y/o modificar este documento
    bajo los términos de la GNU Free Documentation License, Version 1.3
    o cualquier versión publicada por la Free Software Foundation;
    sin Secciones Invariantes y sin Textos de Portada o Contraportada.
    Una copia de la licencia está incluida en \href{https://www.gnu.org/copyleft/fdl.html}{GNU Free Documentation License}.

\href{https://www.gnu.org/copyleft/fdl.html}{\includegraphics[width=.9\linewidth]{https://upload.wikimedia.org/wikipedia/commons/thumb/4/42/GFDL_Logo.svg/200px-GFDL_Logo.svg.png}}
% Emacs 24.3.1 (Org mode 8.3beta)
\end{document}
