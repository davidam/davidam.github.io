% Created 2013-08-18 dom 12:11
\documentclass[11pt]{article}
\usepackage[utf8]{inputenc}
\usepackage[T1]{fontenc}
\usepackage{fixltx2e}
\usepackage{graphicx}
\usepackage{longtable}
\usepackage{float}
\usepackage{wrapfig}
\usepackage[normalem]{ulem}
\usepackage{textcomp}
\usepackage{marvosym}
\usepackage{wasysym}
\usepackage{latexsym}
\usepackage{amssymb}
\usepackage{amstext}
\usepackage{hyperref}
\tolerance=1000
\author{David Arroyo Menéndez}
\date{\today}
\title{Sobre la Autoridad}
\hypersetup{
  pdfkeywords={},
  pdfsubject={},
  pdfcreator={Emacs 24.2.1 (Org mode 8.0.7)}}
\begin{document}

\maketitle
\tableofcontents

\href{http://www.davidam.com}{Volver a davidam.com}

Hoy me gustaría reflexionar sobre las diferentes formas de autoridad
que he encontrado en la sociedad y para que sirve cada una.

\section{La Autoridad Física o Militar}
\label{sec-1}

Esta es la forma de autoridad más evidente el pueblo que somete a otro
por tener mejores armas, el hombre que pega a su mujer, el mayor que
pega al menor, el policía que pega al manifestante, etc. Todos nos
hemos visto sometidos a esta autoridad en algún momento de nuestras
vidas y, a veces, aunque no se ejerza queda latente que quien tiene
mejor armamento o fuerza puede pegar.

Los movimientos pacifistas tienen la convicción de que esta forma de
autoridad es inmoral y, así desde este punto de vista, puede tratar de
ser vencida. Esto es en parte cierto, sin embargo, es necesario que
otras autoridades como la económica que apoyen el punto de vista moral
\section{La Autoridad Estructural}
\label{sec-2}

La forma autoridad que vemos más de cerca diariamente es la
estructural, todo el mundo critica las dictaduras militares como
formas opresivas, sin embargo, lo cotidiano es recibir o dar órdenes
en una empresa por el hecho de que se posee un cargo estructural que
puede ser mantenido por amiguismos, o por cualquier otra razón,
algunas mejores.

De hecho, en las familias está el hecho de que tenemos que obedecer a
nuestros padres y esto va desapareciendo en el momento en el que
mejoramos nuestra autoridad económica y de otro tipo.

Luego está el tema de la política remunerada que nos golpea con sus
comunicados en medios de comunicación de masas y con sus leyes que
obligan a obedecer mediante la autoridad física.

La autoridad estructural puede ser eliminada por dos vías
deslegitimarla por la vía moral y crear estructuras dónde se elimine.
\section{La Autoridad de la Popularidad}
\label{sec-3}

Las personas que tienen popularidad tienen también una forma de
autoridad esta no es tanto de dar órdenes, sino que sus ideas o su
forma de actuar se popularizan porque ellos son populares, a veces,
uno recibe popularidad de manera honesta, como publicar un artículo
científico, o publicar software libre, pero también se puede ver el
camino de recibir popularidad tras un escándalo sexual, un crimen, un
atentado, etc. 

Esta forma de autoridad es muchas veces impulsada por la dialéctica o
el márketing que para mí no es tanto una forma de autoridad, sino una
característica de la propia popularidad que provoca determinados
sentimientos a quien es dirigida.

Como consumidores de información debemos ser sensatos a la hora de a
quien dar popularidad y estaría bien minimizar las fuentes de
información que dan popularidad a cuestiones poco éticas (por ej: no
pasarse el día leyendo prensa del corazón) y ser críticos en cuanto a
la veracidad de las fuentes (por ej: cada publicación tiene una
ideología).

Evidentemente, hay un trasvase claro de la autoridad económica a la
autoridad de la popularidad (ej: uno puede invertir dinero en
anuncios, por ejemplo).
\section{La Autoridad Técnica}
\label{sec-4}

La autoridad técnica es la capacidad de hacer tareas prácticas que
satisfagan de alguna manera el bien común desde construir un puente, a
limpiar zapatos, o programar software. Con esta forma de autoridad uno
se puede ganar un puesto de trabajo de una manera digna.
\section{La Autoridad Moral}
\label{sec-5}

La autoridad moral es tener actos y discursos que si todo el mundo los
repitiera se aumentaría bastante la felicidad colectiva. 

Esto tiene bastantes resbalones, ya que no todo el mundo tiene
energías o conocimientos para hacer las mismas cosas, pero sí hay
actitudes más ó menos éticas en cada una de las profesiones y hay
profesiones menos éticas que otras. Por ejemplo, un psiquiatra puede
utilizar su autoridad técnica para ayudar a una persona a salir de una
situación mentalmente difícil, o para encerrarla para siempre y, un
programador puede utilizar su capacidad para liberar mentes, o para
poseerlas.
\section{La Autoridad Económica}
\label{sec-6}

La autoridad económica se basa en tener dinero o posesiones
materiales. En la sociedad actual las desigualdades económicas baten
récords históricos y es la principal forma de dominación actual. Si
bien no se han abandonado otras viejas fórmulas como la autoridad
esctructural.

Para vivir en la sociedad actual se necesita una economía que te
permita tener casa, comida y salud. Hay economías que subsisten sin
dinero, bien en sociedades primitivas, bien en sociedades anarquistas.
\section{La Creatividad}
\label{sec-7}

Quienes poseen creatividad tienen capacidad de pensar ideas diferentes
a las corrientes y esto puede generar cambios de estructuras y de
poder, por ello mismo se podría ver como una especie de autoridad, si
bien hay que compaginarla con otras formas de autoridad para que
llegue a implementarse, como una cierta autoridad económica (para
financiar la idea), una autoridad técnica (para implementarla) y una
popularidad para que la idea se propague.
\section{Licencia}
\label{sec-8}

Este documento está bajo una \href{http://creativecommons.org/licenses/by-nd/3.0/es/deed}{Licencia Creative Commons Atribución-SinDerivadas 3.0}

\href{http://creativecommons.org/licenses/by-nd/3.0/es/deed}{\includegraphics[width=.9\linewidth]{http://i.creativecommons.org/l/by-nd/3.0/80x15.png}}
% Emacs 24.2.1 (Org mode 8.0.7)
\end{document}
