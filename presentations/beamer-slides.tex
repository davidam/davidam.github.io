% Created 2019-03-06 mié 19:26
\documentclass[unicode,presentation,c,squeeze,shrink,10pt]{beamer}
\usepackage[utf8]{inputenc}
\usepackage[T1]{fontenc}
\usepackage{fixltx2e}
\usepackage{graphicx}
\usepackage{longtable}
\usepackage{float}
\usepackage{wrapfig}
\usepackage{rotating}
\usepackage[normalem]{ulem}
\usepackage{amsmath}
\usepackage{textcomp}
\usepackage{marvosym}
\usepackage{wasysym}
\usepackage{amssymb}
\usepackage{hyperref}
\tolerance=1000
\AtBeginSection[]{\begin{frame}<beamer>\frametitle{Outline}\tableofcontents[currentsection]\end{frame}}
\usetheme{KansaiDebian}
\author{David Arroyo Menéndez}
\date{2017-01-29 16:45:57}
\title{Orgmode \& Beamer}
\hypersetup{
  pdfkeywords={},
  pdfsubject={},
  pdfcreator={Emacs 24.5.1 (Org mode 8.2.10)}}
\begin{document}

\maketitle

\section{Introduction from files}
\label{sec-1}
\begin{frame}[fragile,label=sec-1-1]{Downloads}
 \begin{verbatim}
$ git clone https://github.com/davidam/davidam.github.io
\end{verbatim}
\end{frame}
\begin{frame}[fragile,label=sec-1-2]{A beamer file is a latex file}
 \begin{verbatim}
$ cd ~/git/davidam.github.io/presentations
$ more beamer.tex
\end{verbatim}
\end{frame}
\begin{frame}[fragile,label=sec-1-3]{An orgmode file can be a beamer file with wiki sintaxis}
 \begin{verbatim}
$ cd ~/git/davidam.github.io/presentations
$ more beamer.org
\end{verbatim}
\end{frame}
\begin{frame}[label=sec-1-4]{Orgmode is for \ldots{}}
\begin{itemize}
\item TODO management in text mode
\item Articles (reproducible research, latex capabilities, nice navigation, \ldots{})
\item Spreadsheet
\item Presentation (beamer)
\item Web Deployment
\item Powerful hiperlinks (RMAIL, BBDB, IRC, \ldots{})
\item In general, easy to extend
\item Ready to use in emacs, vim, github, \ldots{}
\end{itemize}
\end{frame}
\section{Why?}
\label{sec-2}
\begin{frame}[label=sec-2-1]{Why is a good idea make slides with latex or wiki sintaxis?}
\begin{itemize}
\item $\boxtimes$ Version Control
\item $\square$ I'm using ssh
\item $\square$ Divide visualization and structure is for winners!
\item $\boxtimes$ Good support for equations
\end{itemize}
\end{frame}

\begin{frame}[label=sec-2-2]{What advantages is adding emacs orgmode to latex beamer}
\begin{itemize}
\item $\boxtimes$ Good navigation
\item $\boxtimes$ Reproducible research
\end{itemize}
\end{frame}
\section{Just do it!}
\label{sec-3}
\begin{frame}[fragile,label=sec-3-1]{I want export my beamer.org}
 \begin{block}{From GNU/Emacs to pdf}
M-x org-beamer-export-to-pdf
\end{block}
\begin{block}{From shell to html}
\begin{verbatim}
$ latex2html beamer.tex
\end{verbatim}
\end{block}
\end{frame}
\begin{frame}[label=sec-3-2]{Themes by example}
Take a look to the \href{http://deic.uab.es/~iblanes/beamer_gallery/index_by_theme.html}{matrix} between beamer color theme and beamer theme.
Some examples
\begin{itemize}
\item $\square$ Madrid
\item $\boxtimes$ beamerthemeKansaiDebianMeeting
\end{itemize}
\end{frame}
\begin{frame}[fragile,label=sec-3-3]{Give me a reproducible calculus}
 The source is:
\begin{verbatim}
(+ 2 2)
\end{verbatim}

The result is:
\begin{verbatim}
4
\end{verbatim}
\end{frame}

\begin{frame}[label=sec-3-4]{Give me documentation}
\begin{itemize}
\item \url{https://orgmode.org/worg/exporters/beamer/tutorial.html}
\item \url{https://orgmode.org/manual/Beamer-export.html}
\end{itemize}
\end{frame}
% Emacs 24.5.1 (Org mode 8.2.10)
\end{document}
