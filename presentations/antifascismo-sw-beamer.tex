% Created 2016-05-22 dom 23:00
\documentclass[bigger]{beamer}
\usepackage[utf8]{inputenc}
\usepackage[T1]{fontenc}
\usepackage{fixltx2e}
\usepackage{graphicx}
\usepackage{longtable}
\usepackage{float}
\usepackage{wrapfig}
\usepackage{rotating}
\usepackage[normalem]{ulem}
\usepackage{amsmath}
\usepackage{textcomp}
\usepackage{marvosym}
\usepackage{wasysym}
\usepackage{amssymb}
\usepackage{hyperref}
\tolerance=1000
\usepackage[spanish]{babel}
\usetheme{Madrid}
\author{David Arroyo Menéndez}
\date{\textit{[2015-10-25]}}
\title{Antifascismo en el Software}
\hypersetup{
  pdfkeywords={antifascismo software libre},
  pdfsubject={Example of using org to create presentations using the beamer exporter},
  pdfcreator={Emacs 24.4.1 (Org mode 8.2.10)}}
\begin{document}

\maketitle
\begin{frame}{Outline}
\tableofcontents
\end{frame}


\section{Introducción}
\label{sec-1}
\begin{frame}[label=sec-1-1]{Introducción: Cultura Material y Software (I)}
\begin{itemize}
\item La cultura material se refiere a valores e ideología que se transmite desde construcciones humanas.

\item El software son cálculos que produce un ordenador.

\item La cultura material de un espacio ayuda a comprender el autoritarismo o antiautoritarismo de una sociedad. Ejemplos.

\item Lo que puedes hacer con el software también ayuda a comprender autoritarismo. ¿Cómo usáis el software?
\end{itemize}
\end{frame}

\begin{frame}[label=sec-1-2]{Introducción: Cultura Material y Software (II)}
En el artículo que nos ocupa nos centraremos en la cuestión de lo que
es fascista y no en el software según diferentes aspectos.
\begin{enumerate}
\item Libertad en el Software
\item Desigualdad Económica
\item Control Social
\item El Papel del Software en el Movimiento Obrero
\end{enumerate}
\end{frame}

\section{Libertad en el Software}
\label{sec-2}
\begin{frame}[label=sec-2-1]{Libertad en el Software (4 libertades)}
\begin{enumerate}
\item Libertad de uso
\item Libertad de copia
\item Libertad de modificación
\item Libertad de redistribuir lo modificado
\end{enumerate}
\end{frame}
\begin{frame}[label=sec-2-2]{Libertad en el Software versus Libertad}
\begin{itemize}
\item En la CNT se aspira a una libertad integral y no parcial del individuo y de la sociedad.
\item Mirar el éxito en el microaspecto es bueno para la mercantilización. Ej: Opensource.
\end{itemize}
\end{frame}

\begin{frame}[label=sec-2-3]{No modificar el código fuente es ideología}
\begin{itemize}
\item El que las personas no deben modificar el código fuente es la ideología que crea tiranías económicas
\end{itemize}
\end{frame}

\section{Desigualdad Económica}
\label{sec-3}
\begin{frame}[label=sec-3-1]{Cifras económicas de algunos personajes}
\begin{center}
\begin{tabular}{rllrll}
Número & Nombre & Dinero(\$) & Edad & País & Compañía\\
1 & Bill Gates & \$ 79.2 & 59 & USA & Microsoft\\
2 & Carlos Slim & \$ 77.1 & 75 & México & Telecom\\
3 & Warren Buffet & \$ 72.1 & 85 & USA & Berkshire\\
4 & Amancio Ortega & \$ 64.5 & 79 & Spain & Zara\\
5 & Larry Ellison & \$ 54.3 & 71 & USA & Oracle\\
15 & Jeff Bezos & \$ 34.8 & 51 & USA & Amazon\\
16 & Mark Zuckerberg & \$ 33.4 & 31 & USA & Facebook\\
19 & Larry Page & \$ 29.7 & 42 & USA & Google\\
20 & Sergey Brin & \$ 29.2 & 42 & USA & Google\\
\end{tabular}
\end{center}

Cifras en miles de millones de dólares. Según Forbes 2015 (visto 2015/10/08)
\end{frame}

\begin{frame}[label=sec-3-2]{Cifras Económicas de algunos países}
\begin{center}
\begin{tabular}{rll}
Número & País & Dólares\\
1 & China & 17 632 014\\
2 & USA & 17 416 253\\
3 & India & 7 277 279\\
4 & Japón & 4 788 033\\
5 & Alemania & 3 621 357\\
16 & España & 1 533 590\\
86 & Líbano & 80 077\\
87 & Lituania & 78 953\\
89 & Costa de Marfil & 71 952\\
90 & Costa Rica & 71 211\\
181 & Tonga & 523\\
182 & Kiribati & 180\\
183 & Tuvalu & 35\\
\end{tabular}
\end{center}

Cifras en millones de dólares. Según FMI 2014 (visto 2015/10/08)
\end{frame}

\begin{frame}[label=sec-3-3]{Cifras de grandes empresas en el mundo del software}
\begin{center}
\begin{tabular}{rlll}
Número & Compañía & País & Dólares\\
12 & Apple & USA & 741.8 B\\
25 & Microsoft & USA & 340.8 B\\
39 & Google & USA & 367.\\
\end{tabular}
\end{center}
\end{frame}


\section{Control Social}
\label{sec-4}
\begin{frame}[label=sec-4-1]{Evitar control social}
\begin{itemize}
\item Primitivismo
\item Software libre en mi hardware, sin espionaje.
\end{itemize}
\end{frame}

\begin{frame}[label=sec-4-2]{Cómo funciona el espionaje masivo}
\begin{itemize}
\item Google: Android + Gmail
\item Facebook
\item Machintosh
\item Microsoft
\end{itemize}
\end{frame}

\section{El papel del software en el movimiento obrero}
\label{sec-5}
\begin{frame}[label=sec-5-1]{El papel del software en el movimiento obrero}
\begin{itemize}
\item La comunicación
\item La apropiación monetaria
\item La dependencia tecnológica
\item La diversidad lingüística
\item El boicot
\item Tecnicidad y certificación en el Software Libre
\end{itemize}
\end{frame}

\section{Diferentes formas de militancia por el antifascismo en el software}
\label{sec-6}
\begin{frame}[label=sec-6-1]{Diferentes formas de militancia por el antifascismo en el software}
\begin{itemize}
\item GNU
\item FSF
\item Opensource
\item Hacktivismo
\item Cultura libre
\item Sindicalismo y anarcosindicalismo
\item Cooperativas
\item Grupos de crackers
\end{itemize}
\end{frame}



\section{Conclusiones}
\label{sec-7}
\begin{frame}[label=sec-7-1]{Conclusión (I): En lo cuantitativo el Software Libre está bien}
\begin{itemize}
\item El software libre se ha ido haciendo fuerte en casi todos los mercados:
\item Supercomputadores
\item Servidores
\item Doméstico
\end{itemize}
\end{frame}
\begin{frame}[label=sec-7-2]{Conclusión (II): En lo cualitativo estamos mal}
\begin{itemize}
\item El control social es mayor
\item Las desigualdades son mayores
\end{itemize}
\end{frame}
% Emacs 24.4.1 (Org mode 8.2.10)
\end{document}
